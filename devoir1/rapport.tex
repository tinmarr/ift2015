\documentclass[12pt]{article}

\usepackage[margin=1in]{geometry}
\setlength{\parindent}{0pt}

\title{Devoir 1}
\author{Martin Chaperot\\20205638}
\date{}

\begin{document}

\maketitle

\section*{2}
\subsection*{1}
Temps: \(O(2^n)\)\\
Espace: \(O(n)\)

Cette fonction est une implementation récursive de la fonction d'Ackermann.

\begin{verbatim}
        (2,2)
        /    \
    (1,2)     (2,1)
    /   \     /   \
(0,2) (1,1) (1,1) (2,0)
        / \
    (0,1) (1,0)
\end{verbatim}

\subsection*{2}
Temps: \(O(n^3)\)\\
Espace: \(O(n^3)\)

Cette fonction est une implémentation de la programmation dynamique
pour résoudre le problème de la construction de la chaîne cible \verb|target| en
utilisant les sous-chaînes fournies dans la liste \verb|pieces|.

\subsection*{3}
Temps: \(O(n^2)\)\\
Espace: \(O(n^2)\)

Cette fonction est une implémentation de la programmation dynamique pour
déterminer s'il est possible de sommer des éléments du tableau \verb|options|
pour atteindre le nombre \verb|target|.

\subsection*{4}
La complexité temporelle augmenterais.

La complexité pour les deux fonctions est \(O(2^n)\). Car pour chacune des fonctions,
nous devons faire des calcules pour chaque element avec chaque autre element.

\section*{3.4}
\subsection*{1}
L'algorithme déplace une section à la fois de Nam à Pam, puis de Pam à Sam,
en respectant l'ordre original. Ce processus se répète jusqu'à ce que toutes
les sections soient déplacées de Nam à Sam.

Complexité temporelle: \(O(n)\)\\
Complexité spatiale: \(O(n)\)

\subsection*{2}
Complexité temporelle: \(O(n)\)\\
Complexité spatiale: \(O(n)\)

\end{document}
